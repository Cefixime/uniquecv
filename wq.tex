\documentclass{uniquecv}

\usepackage{fontawesome}

% ----------------------------------------------------------------------------- %

\begin{document}

\name{魏强}

\medskip

\basicinfo{
  \faPhone ~ (+86) 158-2684-8894
  \textperiodcentered\
  \faEnvelope ~ 1627889449@qq.com
  \textperiodcentered\
  \faGithub ~ github.com/Cefixime
}


% ----------------------------------------------------------------------------- %

\section{教育背景}
\dateditem{\textbf{华中科技大学} \quad 计算机科学与技术学院 \quad 本科}{2018年 -- 2022年}
\dateditem{校三好学生}{2019年、2020年}
成绩:年级前5\% \quad 英语:CET6


% ----------------------------------------------------------------------------- %

\section{专业技能}
\smallskip
\begin{itemize}
\item 熟悉C语言,了解和使用过C++、Golang
\item 熟悉大部分数据结构和算法
\item 对计算机网络和操作系统有一定的了解
\end{itemize}

% ----------------------------------------------------------------------------- %

\section{获奖情况}
\datedaward{Finalist}{美国大学生数学建模竞赛(MCM/ICM)}{2020年03月}
\datedaward{全国三等奖}{\small{2020中国高校计算机大赛·华为云大数据挑战赛}}{2020年06月}
\medskip

% ----------------------------------------------------------------------------- %

\section{项目经历}

\datedproject{TinyKV}{开源项目}{2020年12月——2021年01月}
\textit{Go、分布式KV数据库}
\vspace{0.4ex}


基于Go语言的分布式 Key-Value 数据库,接近TiKV的实现。在基于已有调度逻辑的基础上
从 0 到 1 实现了一个完整可用的分布式 KV 服务。
\begin{itemize}
  \item 基于Raft协议实现的多副本高可用KV server
  \item 基于 Percolator 模型实现分布式事务
  \item 实现了多机节点的数据均衡调度
  \item 通过了绝大多数测试点
\end{itemize}
% ---
\datedproject{“Huster爱选课”选课系统}{团队项目}{2020年12月}
\textit{Python、Web应用}
\vspace{0.4ex}

项目实现了一个选课系统,能够模拟学生选课、教师发布课程和审批学生选课功能。
\begin{itemize}
  \item 采用Flask框架进行快速开发
  \item 数据库使用sqlite,ORM框架使用SQLAlchemy
  \item 担任后端开发角色,完成了业务逻辑处理模块和数据库模块
\end{itemize}

\medskip
% ---
\datedproject{SAT求解器}{个人项目}{2020年01月}
\textit{C、数据结构}
\vspace{0.4ex}

使用C语言实现的一个简易N-SAT问题的求解器
\begin{itemize}
  \item 运用CDCL(冲突子句学习)方法加快回溯
  \item 搜索过程使用了链式结构存储减小内存消耗
  \item 实现了朴素算法10x加速比
\end{itemize}
\medskip
% ---


% ----------------------------------------------------------------------------- %
\end{document}

